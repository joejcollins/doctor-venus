\documentclass[10pt, twoside, openright, a5paper]{memoir}
\usepackage[paperwidth=148mm, paperheight=210mm, bindingoffset=.75in]{geometry}
\usepackage[latin1]{inputenc}

%% 
% Set the font and adjust a few sizes
%%
\renewcommand{\familydefault}{cmss}
\newcommand{\smaller}{\small}
\raggedright 

%\title{TITLE OF BOOK}
%\author{NAME OF AUTHOR}
%\date{} % Delete this line to display the current date

%% BEGIN TITLE

\makeatletter
\def\maketitle{%
  \null
  \thispagestyle{empty}%
  \vfill
  \begin{center}\leavevmode
    \normalfont
    {\LARGE\raggedleft \@author\par}%
    \hrulefill\par
    {\huge\raggedright \@title\par}%
    \vskip 1cm
%    {\Large \@date\par}%
  \end{center}%
  \vfill
  \null
  \cleardoublepage
  }
\makeatother
\author{NAME OF AUTHOR}
\author{NAME OF AUTHOR}
\title{TITLE OF BOOK}
\date{}










%%% BEGIN DOCUMENT

\begin{document}

\let\cleardoublepage\clearpage


\maketitle


\frontmatter

\null\vfill

\begin{flushleft}
\textit{NAME OF BOOK}


\copyright\ COPYRIGHT INFO


ISBN--INFO

ISBN--13: 
\bigskip



\end{flushleft}
\let\cleardoublepage\clearpage

\mainmatter
\sloppy

\chapter{A Taste of Mindfulness}

\section{What is Mindfulness}

To a greater or lesser extent we're all starting out with some habits that may not serve us very well.

\begin{itemize}
  \item We tend to be a very `thinking' society. Often we believe problems should be able to be solved if only we can give them enough thought and attention.
  \item We tend to have an aversion to negative moods, and a strong attachment to feeling happy.
  \item We're often comparing our current state to what we think our 'ideal' state should be\ldots hence judging our experience to be good or bad.
  \item And we're very often living our lives on automatic pilot.
\end{itemize}

In the course, we first aim to learn to pay attention, and when we can see what is going on in our minds we can learn to perhaps respond differently to our thoughts and feelings, or at least to have the choice in how to respond. A bit like swimming, first we have to learn how to stay afloat for a reasonable period, and then we can learn different strokes. We are not aiming to become relaxed or happy, we are aiming to become aware of what is, and to become more accepting of it. And oddly enough, we may find happiness and relaxation easier to find as a result of being more aware and accepting.

\section{Automatic pilot}

In a car we can drive for miles without really being aware of what we are doing. 
In the same way we may not really be present in a moment to moment sense, 
for much of our lives. 
We can be miles away without being aware of it.

In this Automatic Pilot state, 
we are much more likely to have our buttons pressed. 
Events around us, or within us, 
such as thoughts, feelings and sensations in the body, of which we may only be dimly aware, 
can trigger old habits of thinking and 
trains of thought that are often unhelpful and can significantly affect our moods and how we relate to our environment and selves.

When we become more aware of our thoughts, 
feelings and sensations in our bodies, 
on a moment to moment basis, we give ourselves the possibility of greater freedom and choice. 
It's no longer inevitable for us to go into the same mental ruts that we've been used to in the past.

So our first aim is to increase our awareness so that we can learn to respond to situations with choice rather than reacting automatically. 
We  do that by practicing to be more aware of where our attention is, 
and later by deliberately changing the focus of attention, 
over and over again. 

As we saw in the Raisin Exercise, paying attention to what is in front of us can change that experience.  
When we are in automatic pilot there is a strong tendency to assume we know what our experience is going to be, 
and therefore to not really be there for it, and not get our `moment's worth'.  Missing the good experiences in our lives means we do not experience our lives as richly as they really might be.  Also, being unaware of the more difficult issues in our lives makes it difficult to take skillful action to improve our lives or situations.

\section{Home Practice}

\begin{itemize}
  \item Sit for 2 minutes a couple of times each day, as we did in the class. 
    There is no agenda and no particular state to be achieved. 
    Simply notice what happens or how you feel for 2 minutes each day.
  \item Make a daily entry in the Pleasant Events Diary.
  \item Eat something mindfully every day in the way you ate the raisin.
\end{itemize}

\end{document}